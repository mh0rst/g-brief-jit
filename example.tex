\documentclass[11pt]{g-brief-jit}
\usepackage[utf8]{inputenc}
\usepackage{ngerman}
\usepackage[normalem]{ulem}
% Pflichtangaben
\Anrede              {Sehr geehrte Damen und Herren}
\Gruss               {Mit freundlichen Grüßen}{1cm}
\Unterschrift        {Max Mustermann}

% Alle Angaben ab hier sind optional und können durch Auskommentieren weggelassen werden

\Name                {Max Mustermann}
\Strasse             {Musterstr. 10}
%\Zusatz              {}
\Ort                 {12345 Musterstadt}
\Land                {}
% Wenn RetourAdresse gesetzt ist, wird diese im Brieffenster gedruckt,
% ansonsten die Adresse aus \Name \Strasse \Ort und \Land
%\RetourAdresse       {}

\Ustidnr	 		 {DE1234567}
\Steuernummer	 	 {12/345/67890}

\Telefonnr           {+49 (0)1234 56789 0}
%\Telefax             {}
%\Telex               {}
\HTTP                {https://www.example.com}
\EMail               {mail@example.com}

\Bank                {Beispielbank}
\BLZ                 {BIC123ASD}
\Konto               {DE121234567801234567890}


%\Postvermerk         {}
\Adresse             {Empfänger GmbH\\
					  Musterstr. 10\\
					  12346 AndereStadt}

\Betreff             {Beispielbetreff}

\Datum               {\today}
\IhrZeichen          {ASD345345}
\IhrSchreiben        {10.10.2010}
\MeinZeichen         {3-45-2134}


%\Anlagen             {}
%\Verteiler           {}

% Markierungen für Adressfenster, Faltungen etc.
%\lochermarke
\faltmarken
\fenstermarken
\trennlinien

% Weitere Optionen
%\klassisch %-> altes Kopfzeilenformat, Betr.: vor Betreff und Unterschrift kursiv und in Klammern
%\unserzeichen %-> "Unser Zeichen" statt "Mein Zeichen"

\begin{document}
\sloppy
\begin{g-brief-jit}
Damit Ihr indess erkennt, woher dieser ganze Irrthum gekommen ist, und weshalb man die Lust anklagt und den Schmerz lobet, so will ich Euch Alles eröffnen und auseinander setzen, was jener Begründer der Wahrheit und gleichsam Baumeister des glücklichen Lebens selbst darüber gesagt hat.

Niemand, sagt er, verschmähe, oder hasse, oder fliehe die Lust als solche, sondern weil grosse Schmerzen ihr folgen, wenn man nicht mit Vernunft ihr nachzugehen verstehe. Ebenso werde der Schmerz als solcher von Niemand geliebt, gesucht und verlangt, sondern weil mitunter solche Zeiten eintreten, dass man mittelst Arbeiten und Schmerzen eine grosse Lust sich zu verschaften suchen müsse.

Um hier gleich bei dem Einfachsten stehen zu bleiben, so würde Niemand von uns anstrengende körperliche Uebungen vornehmen, wenn er nicht einen Vortheil davon erwartete. Wer dürfte aber wohl Den tadeln, der nach einer Lust verlangt, welcher keine Unannehmlichkeit folgt, oder der einem Schmerze ausweicht, aus dem keine Lust hervorgeht?

Damit Ihr indess erkennt, woher dieser ganze Irrthum gekommen ist, und weshalb man die Lust anklagt und den Schmerz lobet, so will ich Euch Alles eröffnen und auseinander setzen, was jener Begründer der Wahrheit und gleichsam Baumeister des glücklichen Lebens selbst darüber gesagt hat.

Niemand, sagt er, verschmähe, oder hasse, oder fliehe die Lust als solche, sondern weil grosse Schmerzen ihr folgen, wenn man nicht mit Vernunft ihr nachzugehen verstehe. Ebenso werde der Schmerz als solcher von Niemand geliebt, gesucht und verlangt, sondern weil mitunter solche Zeiten eintreten, dass man mittelst Arbeiten und Schmerzen eine grosse Lust sich zu verschaften suchen müsse.

Um hier gleich bei dem Einfachsten stehen zu bleiben, so würde Niemand von uns anstrengende körperliche Uebungen vornehmen, wenn er nicht einen Vortheil davon erwartete. Wer dürfte aber wohl Den tadeln, der nach einer Lust verlangt, welcher keine Unannehmlichkeit folgt, oder der einem Schmerze ausweicht, aus dem keine Lust hervorgeht?

Damit Ihr indess erkennt, woher dieser ganze Irrthum gekommen ist, und weshalb man die Lust anklagt und den Schmerz lobet, so will ich Euch Alles eröffnen und auseinander setzen, was jener Begründer der Wahrheit und gleichsam Baumeister des glücklichen Lebens selbst darüber gesagt hat.

Niemand, sagt er, verschmähe, oder hasse, oder fliehe die Lust als solche, sondern weil grosse Schmerzen ihr folgen, wenn man nicht mit Vernunft ihr nachzugehen verstehe. Ebenso werde der Schmerz als solcher von Niemand geliebt, gesucht und verlangt, sondern weil mitunter solche Zeiten eintreten, dass man mittelst Arbeiten und Schmerzen eine grosse Lust sich zu verschaften suchen müsse.

Um hier gleich bei dem Einfachsten stehen zu bleiben, so würde Niemand von uns anstrengende körperliche Uebungen vornehmen, wenn er nicht einen Vortheil davon erwartete. Wer dürfte aber wohl Den tadeln, der nach einer Lust verlangt, welcher keine Unannehmlichkeit folgt, oder der einem Schmerze ausweicht, aus dem keine Lust hervorgeht?

\end{g-brief-jit}
\end{document}


\endinput